% Abstract, in Greek

\begin{abstractgr}
  Η αυξημένη χρήση των Εικονικών Μηχανών στις διάφορες υπηρεσίες νέφους οδήγησε στη δημιουργία ενός μεγάλου αριθμού αρχείων εικόνων και στιγμιοτύπων εικονικών μηχανών. Γεννήθηκε έτσι η ανάγκη για ένα αξιόπιστο και αποδοτικό τρόπο συγχρονισμού των αρχείων αυτών μεταξύ διαφορετικών υπολογιστών. Υπάρχουν ήδη λογισμικά που υλοποιούν συγχρονισμό αρχείων, αλλά κανένα δεν είναι φτιαγμένο ειδικά γι' αυτό το σκοπό. Μελετώντας και κατανοώντας τα ιδιαίτερα χαρακτηριστικά αυτής της μορφής αρχείων, δηλαδή το γεγονός πως είναι μεγάλα σε μέγεθος και έχουν πολλά κοινά δεδομένα μεταξύ τους, μας επιτρέπει να βελτιστοποιήσουμε τη διαδικασία συγχρονισμού τους.

  Ο στόχος αυτής της διπλωματικής εργασίας είναι να παρουσιάσει τη σχεδίαση μίας βιβλιοθήκης στη γλώσσα Python, για το συγχρονισμό μεγάλων ομοιόμορφων αρχείων με χρήση υπηρεσιών αποθηκευτικού νέφους. Μελετάμε τις υπάρχουσες υλοποιήσεις για συγχρονισμό αρχείων, κατανοούμε τις σχεδιαστικές επιλογές πίσω από αυτά και τα επεκτείνουμε περαιτέρω, με νέες βελτιστοποιήσεις. Προτείνουμε ένα αλγόριθμο συγχρονισμού που ανιχνεύει και χειρίζεται ενημερώσεις σε αρχεία αποδοτικά και αξιόπιστα. Προτείνουμε επίσης τη χρήση αφαιρετικών κλάσεων για την αναπαράσταση των αρχείων, των τοπικών καταλόγων αρχείων και των προγραμματιστικών διεπαφών εφαρμογών (API) των υπηρεσιών αποθηκευτικού νέφους. Τα API που εκθέτους οι παραπάνω αφαιρετικές κλάσεις επιτρέπουν μεγαλύτερη ευελιξία στη βιβλιοθήκη, δίνοντάς του τη δυνατότητα να λειτουργήσει πάνω σε διαφορετικά λειτουργικά συστήμάτα και υπηρεσίες αποθηκευτικού νέφους.

  Μετά την παρουσίαση της αρχικής σχεδίασης, προτείνουμε και υλοποιούμε διάφορες βελτιστοποιήσεις που βελτιώνουν περαιτέρω την απόδοση της διαδικασίας συγχρονισμού και πραγματοποιούμε συγκριτικές αξιολογήσεις ώστε να μετρήσουμε την επίδρασή τους στο χρόνο εκτέλεσης. Η χρήση νημάτων για την ταυτόχρονη αποστολή αιτημάτων στον απομακρυσμένο εξυπηρετητή μειώνει την επίδραση της καθυστέρησης του δικτύου, ενώ η χρήση μηχανισμών παρακολούθησης του καταλόγου αρχείων (όπως το inotify) έχει ως αποτέλεσμα την ταχεία και αποδοτική ανίχνευση των τροποποιημένων αρχείων. Εστιάζοντας περισσότερο στο σενάριο χρήσης των μεγάλων ομοιόμορφων αρχείων, προτείνουμε την τοπική αποθήκευση των block των αρχείων, ώστε να μεταφορτώνονται μόνο τα κομμάτια που διαφέρουν από τον εξυπηρετητή, κατί που προσφέρει αξιοσημείωτη βελτίωση στο χρόνο μεταφόρτωσης των αρχείων. Τέλος, προκειμένου να αντιμετωπίσουμε τις ανάγκες επιπλέον αποθηκευτικού χώρου που εισήγαγε η τελευταία βελτιστοποίηση, προτείνουμε τη χρήση ενός μηχανισμού συστήματος αρχείων σε περιβάλλον χρήστη (FUSE) που θα επιτρέπει την εικονική δημιουργία και πρόσβαση στα αρχεία, ενώ κάθε μοναδικό block αρχείου θα απόθηκεύεται μία φορά, και ας είναι κοινόχρηστο από περισσότερα αρχεία.

  Στα τελευταία μέρη της διπλωματικής εργασίας, συγκρίνουμε τα προτεινόμενα στοιχεία και την απόδοσή της βιβλιοθήκης με αυτά διαφόρων δημοφιλών λογισμικών και πακέτων συγχρονισμού αρχείων και έπειτα κρίνουμε την καταλληλότητα του καθενός για το σενάριο χρήσης που περιγράφηκε. Προτείνουμε μερικές επιπλέον βελτιστοποιήσεις στη διαδικασία συχρονισμού, οι οποίες έχουν προγραμματιστεί για το μέλλον, αλλά δεν έχουν ακόμh υλοποιηθεί.

\begin{keywordsgr}
    Αποθηκευτικό Νέφος, Υπηρεσίες Νέφους, Εικονικές Μηχανές, Συγχρονισμός αρχείων, Μεγάλα Ομοιόμορφα Αρχεία, Προγραμματιστικές Διεπαφές Εφαρμογών
\end{keywordsgr}
\end{abstractgr}

\onehalfspacing
% Abstract, in English
\begin{abstracten}
  The increased use of Virtual Machines in cloud service infrastructures has resulted in a large volume of disk image and snapshot files. As a result, a reliable and efficient way of synchronising those files between different computers is needed. Software applications that achieve file synchronisation already exist, but none is tailored specifically for this task. Understanding the special characteristics of the files in question, the fact that they are large in size and have most of their data in commmon, allows us optimise the synchronisation process for that use case.

  The aim of this dissertation is to present the design of a synchronisation framework for large similar files, using cloud storage services, written in Python. We study existing implementations of file syncing, understand the underlying design choices and make further improvements on them. We propose a synchronisation algorithm that reliably and efficiently detects and handles updates. We also propose the use of abstract classes to represent files, local directories and cloud storage service APIs. The APIs exposed by those classes allow more flexibility to the framework, so it can operate over different OSs and cloud storage services.

  After the presentation of the initial design, we propose and implement several optimisations that further improve the performance of the synchronisation process and benchmark their effects on the execution time. The use of threads to concurrently request resources from a remote server reduces the effect of network latency and the use of directory monitoring mechanisms (such as inotify) results in fast and efficient discovery of modified files. Further focusing on the use case of large, similar files, we propose local storage of the files' blocks so only the parts that differ can be downloaded from the server, which boasts a significant improvement in download times. Finally, in order to alleviate the storage space needs that the last improvement introduces, we propose the use of a Filesystem in Userspace (FUSE) mechanism to virtually create and access files, while storing each shared block only once.

  In the final parts of this dissertation, we compare the proposed features and performance with that of several synchronisation software and packages and discuss their suitability for the use case described. We also propose further improvements in the synchronisation process that have been planned but not yet thoroughly designed and implemented.

\begin{keywordsen}
    Cloud storage, Cloud services, Virtual Machines, File synchronisation, Large Similar Files, Application Programming Interfaces
\end{keywordsen}
\end{abstracten}


% Acknowledgements
\begin{acknowledgementsgr}
  Η παρούσα διπλωματική εργασία σημαίνει την ολοκλήρωση ενός σημαντικού κεφαλαίου της ακαδημαϊκής μου πορείας. Θα ήθελα στο σημείο αυτό να ευχαριστήσω ορισμένους ανθρώπους που με βοήθησαν στη διαδρομή αυτή.

  Αρχικά θα ήθελα να ευχαριστήσω τον καθηγητή μου Νεκτάριο Κοζύρη, που μου επέτρεψε να ασχοληθώ με ένα σύγχρονο θέμα που παρουσιάζει ιδιαίτερο πρακτικό ενδιαφέρον. Οφείλω επίσης ένα μεγάλο ευχαριστώ στο Δρα Βαγγέλη Κούκη, για την ομαλή μας συνεργασία, την επιστημονική καθοδήγηση και την ενθάρρυνση που μου προσέφερε κατά τη διάρκεια της διπλωματικής εργασίας, όπως επίσης και για το γεγονός πως με τις διαλέξεις του ενίσχυσε σημαντικά το ενδιαφέρον μου σε αυτό τον τομέα.

  Θα ήθελα επίσης να ευχαριστήσω τους συμφοιτητές, συνεργάτες και φίλους που ομόρφυναν σημαντικά τα χρόνια της φοίτησής μου, Ελένη, Γρηγόρη, Ορέστη Α., Διονύση, Αλέξανδρο, Στέργιο, Σοφία, Ορέστη Β., Δημήτρη, Νίκο, Θάλεια, Μανώλη, Λυδία και Ειρήνη όπως και τους φίλους μου Γιώργο, Έλενα, Πωλίνα, Σπύρο, Νίκο και Τίνα και άρκετούς ακόμη που ίσως αυτή τη στιγμή να μου διαφεύγουν.

  Τέλος, θα ήθελα να ευχαριστήσω τους γονείς μου, Αλέξανδρο και Χαρά, και τον αδερφό μου, Σταύρο, για τη συνεχή υποστήριξη και συμπαράσταση που μου προσέφεραν εώς τώρα.
\end{acknowledgementsgr}
