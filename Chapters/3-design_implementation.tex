% !TEX root = ../thesis.tex

\chapter{Design \& Implementation}

% Θα πρότεινα αυτό το κεφάλαιο να είναι Design and Implementation για τη
% βασική προσέγγιση. Πρώτα ο αλγόριθμος, μετά ο σκελετός σε Python. Πρώτα
% ο αλγόριθμος που θα μπορεί να αναφερθεί σε πράγματα που ήδη έχεις
% καλύψει από το κεφ. 2 [π.χ. τι είναι ETag; πώς χρησιμοποιώ ένα API στο
% οποίο κάθε αρχείο έχει το δικό του URL κλπ], και μετά προχωράς στην
% περιγραφή του βασικού Pythonικού σκελετού.

\section{Syncing Algorithm}
  Mauris id lobortis quam, vitae convallis ipsum. Etiam eget hendrerit purus.
  Aenean ante orci, porta in turpis at, congue posuere dui.

\section{Basic Classes}
  \subsection{FileStat}
    Nulla lectus justo, vulputate non euismod quis, sagittis sit amet sem.
    \subsubsection{Path hash algorithm selection}
      Donec euismod ante non felis condimentum efficitur. Nunc vel pretium diam.
  \subsection{StateDB}
    Nulla lectus justo, vulputate non euismod quis, sagittis sit amet sem.
  \subsection{LocalDirectory}
    Donec euismod ante non felis condimentum efficitur. Nunc vel pretium diam.
  \subsection{CloudClient}
    Lorem ipsum dolor sit amet, consectetur adipiscing elit.
