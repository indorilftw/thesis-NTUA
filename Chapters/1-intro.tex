% !TEX root = ../thesis.tex

\chapter{Εισαγωγή}
% Αυτό το κεφάλαιο είναι το ελληνικό υποκατάστατο του πλήρους κειμένου;
% Αν ναι, θα πρότεινα να ονομαστεί "Περίληψη", από το [extended] Abstract.
% Τα υπο-κεφάλαιά του θα είναι 1-1 τα κεφάλαια του πρωτότυπου αγγλικού
% κειμένου. Θα πρότεινα να μην μας απασχολήσει τώρα. Θα το γράψεις προς το
% τέλος, όταν ήδη το κείμενο έχει αποκτήσει μορφή.
% Έχω κάνει κάτι ανάλογο κι εδώ, αυτό είδα και προτείνω να ακολουθήσουμε
% τώρα:
% http://www.cslab.ece.ntua.gr/~vkoukis/files/vkoukis-phd.pdf

\section{Κίνητρο}
  TODO: Κίνητρο

\section{Κύρια σημεία της εργασίας}
  TODO: Κύρια σημεία της εργασίας

\section{Οργάνωση κειμένου}
  TODO: Οργάνωση κειμένου

\section{Συνοπτική παρουσίαση του framework}
  TODO: Σύνοψη framework

\section{Συνοπτική παρουσίαση των πειραματικών αποτελεσμάτων}
  TODO: Σύνοψη πειραματικών αποτελεσμάτων

\setcounter{chapter}{0}

\chapter{Introduction}

\section{Motivation}
  TODO: Motivation

\section{Thesis contribution}
  TODO: Thesis contribution

\section{Chapter outline}
  TODO: Chapter outline

\section{Brief description of the framework}
  TODO: Brief framework description
